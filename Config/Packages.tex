% Anpassung des Seitenlayouts 
\usepackage[
   automark, 					% Kapitelangaben in Kopfzeile automatisch erstellen
   headsepline, 				% Trennlinie unter Kopfzeile
   ilines 						% Trennlinie linksbündig ausrichten
]{scrlayer-scrpage}
%
% Anpassung an Landessprache 
%\usepackage[ngerman]{babel}
\usepackage[english]{babel}
%
% Encoding 
%\usepackage{fontspec}
%
% Euro-Zeichen etc.
\usepackage{textcomp}
%
% Subfigures
\usepackage{subfigure}
%
% Wrapfigures
\usepackage{wrapfig} 
%
% Mathe
\usepackage[intlimits]{amsmath}
\usepackage{amssymb}
\usepackage{empheq}
\usepackage{nicefrac}
\usepackage{upgreek}
\usepackage[
locale = US,
output-decimal-marker = {.}
]{siunitx}
\sisetup{locale=US, output-decimal-marker = {.}}
%
% Index
\usepackage{makeidx}
%
% Einfache Definition der Zeilenabsände und Seitenränder etc.
\usepackage{setspace}
\usepackage{geometry}
%
% Abkürzungsverzeichnis 
\usepackage[printonlyused]{acronym}
%
% zum Umfliegen von Bildern 
\usepackage{floatrow}
%
% einfachere Farbendefinition
\usepackage[usenames,dvipsnames]{xcolor}
%
% Programmcode 
\usepackage{listings}
%
% URL verlinken 
\usepackage{url}
%
% PDF-Optionen 
\usepackage[
   bookmarks,
   bookmarksopen=true,
   %diese Farbdefinitionen zeichnen Links im PDF farblich aus (zum drucken auskommentieren)
   colorlinks=true,
   linkcolor=blue, 			% einfache interne Verknüpfungen
   anchorcolor=black,			% Ankertext
   citecolor=blue, 			% Verweise auf Literaturverzeichnis im Text
   filecolor=magenta, 			% Verknüpfungen, die lokale Dateien öffnen
   menucolor=red, 				% Acrobat-Menüpunkte
   urlcolor=cyan,
   %
   plainpages=false, 			% zur korrekten Erstellung der Bookmarks
   pdfpagelabels, 				% zur korrekten Erstellung der Bookmarks
   hypertexnames=false, 		% zur korrekten Erstellung der Bookmarks
   linktocpage 				% Seitenzahlen anstatt Text im Inhaltsverzeichnis verlinken
]{hyperref}
%
% fortlaufendes Durchnummerieren der Fußnoten 
\usepackage{chngcntr}
%
% Tabellen 
\usepackage{tabularx}
\usepackage{multirow}
\usepackage{longtable}
\usepackage{colortbl}
\usepackage{array}
\usepackage{ragged2e}
\usepackage{lscape}
%
% bei der Definition eigener Befehle benötigt
\usepackage{ifthen}
%
% definiert u.a. die Befehle \todo und \listoftodos
\usepackage{todonotes}
%
% römische Zahlen
\usepackage{romannum}
%
% zum benutzen von toprule
\usepackage{booktabs}
%
% Rahmen
\usepackage{framed, color}
\usepackage[most]{tcolorbox}
%
% Biblioraphie
\usepackage{csquotes}
\usepackage[style=numeric, sorting=none, backend=biber]{biblatex}
%
% Tikz
\usepackage{color}
\usepackage{transparent}
\usepackage{pgfplots}
%
% Caption
\usepackage{caption}
%
% Platz hinter Komma
%\usepackage{ziffer}
%
% Tabellen farben
\usepackage{colortbl}
\usepackage{hhline}
%
% package pgf plots
\usepackage{pgfplots}
%
% bra-ket / dirac notation
\usepackage{braket}