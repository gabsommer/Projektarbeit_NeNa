
\chapter{Finding Minima of Neon Replacements}
\label{chap:Erstes Kapitel}
\section{Neon interaction LAMMPS setup}
\subsection{\ac{HF} equations}
We start with the pair potentials of Ne-Ne, Ne-Na and Na-Na to mimic the interaction of long distance van der Vaals + short distance repulsion forces. The pair potentials have been calculated using a Hartree Fock self consitent field cycle by variyng the atom distance. Now selfconsitently solving the equations:\\
\begin{align}
	\left[-\frac{\hbar^2}{2m}\vec{\nabla}^2-\frac{Ze^2}{4\pi\varepsilon_0}\sum_{\nu,\sigma}^{(\mu\sigma)\neq(\nu\sigma)}-\frac{e^2}{4\varepsilon_0}\iiint_{\mathbb{R}^3}\frac{\varphi_{\nu\sigma'}(\tilde{\vec{r}})}{\|\vec{r}-\tilde{\vec{r}}\|}\mathrm{d}^3\tilde{\vec{r}}-\hat{A}_{\mu\sigma}(\vec{r})\right]\varphi_{\mu\sigma}(\vec{r})=\varepsilon_{\mu\sigma}\varphi_{\mu\sigma}(\vec{r}),
\end{align}
with $\hat{A}_{\mu\sigma}$ being the exchange correlation term leads to the eigenenergies $\varepsilon_{\mu\sigma}$ Since \ac{HF} 
minimizes a slater determinant, the energies correspond to the factors of the slater product state, meaning the systems energy is the sum of the energy of the single electrons. From that, to get the actual energy of the whole system, one needs to remove inter-electron repulsion and exchange correlation from the sum of all eigenenergies $\varepsilon_{\mu\sigma}$:
\begin{align}
	E_{HF}^0 = \sum_{\mu,\sigma}\varepsilon_{\mu\sigma}-\frac{1}{2}\sum_{\substack{\mu,\nu\\\sigma,\sigma'}}^{(\mu\sigma\neq\nu\sigma)}\left[C_{\mu\sigma}^{\nu\sigma'}-A_{\mu\sigma}^{\nu\sigma}\delta_{\sigma\sigma'}\right]
\end{align}
with $C_{\mu\sigma}^{\nu\sigma'},A_{\mu\sigma}^{\nu\sigma}$ being defined as:
\begin{align}
	C_{\mu \sigma}^{v \sigma^{\prime}}&=\frac{e^2}{4 \pi \varepsilon_0} \iint  \frac{\left|\varphi_{\mu \sigma}(r)\right|^2\left|\varphi_{v \sigma^{\prime}}\left(r^{\prime}\right)\right|^2}{\left|r-r^{\prime}\right|}\mathrm{~d}^3 r \mathrm{~d}^3 r^{\prime},
	\\
	A_{\mu \sigma}^{v \sigma}&=\frac{e^2}{4 \pi \varepsilon_0} \iint \frac{\varphi_{\mu \sigma}^*(r) \varphi_{v \sigma}^*\left(r^{\prime}\right) \varphi_{\mu \sigma}\left(r^{\prime}\right) \varphi_{v \sigma}(r)}{\left|r-r^{\prime}\right|} \mathrm{~d}^3 r \mathrm{~d}^3 r^{\prime}.
\end{align}
Since solving the \ac{HF} equations in real space is too costly one restricts the variations to coefficients of a superposition of fixed basisvectors. These lead to the Roothaan-Hall equations, which are a discrete Matrix representation of the \ac{HF} equations:
\begin{align}
	\mathbf{F C}=\mathbf{S C} \epsilon,
\end{align} 

with $\epsilon$ being a diagonal matrix of single particle energies on the diagonal, $\mathbf{S}$ the basis overlap matrix, $\mathbf{C}$ the coefficient matrix and finally $\mathbf{F}$ the Fock operator.
\subsection{\ac{BSSE}}
Now the the superpositions of a fixed basis leads to the \ac{BSSE}. These errors are being corrected for by using 'ghost calculations', essentially the same calculation with one atom being removed (i.e. the extended two atom centered basis stays the same). This means the basis will have access to the second electron states, but the nucleus and additional electrons are not accounted for. These 'ghost energies' will then subsequently be removed from the \ac{HF}
approximation to correct for the \ac{BSSE}.
This is know as the counterpoise‐corrected interaction energy formula or Boys–Bernardi method: %TODO citation
\begin{align}
	E_{\text {int }}^{\text {CP }}(R)=E_{A B}(R)-E_A^{\text {ghost }}(R)-E_B^{\text {ghost }}(R).
\end{align}
\subsection{Moeller-Plesset Perturbation}
-go form pair HF to moller plesset
\subsection{Interpolation}

\begin{figure}[h!]
	\centering
	\includegraphics[scale = 0.7]{Inhalt/Bilder/pairpotential.png}
	\caption{Plots of interpolations of pair potentials of Ne-Ne, Na-Na, Ne-Na calculated with Moeller Plesset 2 perturabtion theory.}
\end{figure}
%TODO Noltig source
-explain exact interpolation\\
-paper citation\\
-which basisi is being used in the roothaan equations \\
-explain setup of calculation with ball being carved out\\
-maybe here explain the nearest neighbours\\
lammps gets linear interpolation\\
-since plot is in priniple linearly interpolated this is the exact pair interactins that lammps will read
\section{Sodium Monomer}
-Not feasible to brute force dimer with 3rd nearest neighbors\\
-explain nearest neighbors\\
-even with full 48 cubic point symmetry of the host of the defect which no structure can exceed ...\\ 
-explain confirmation of algorithm with brute force for mono sodium\\
-explain how plot is created with sweeps\\
-more complicated symmetries\\
-compare figures\\
-calculation for dimer\\
-discuss noteworthy structure (e.g. inner shell carved out)\\
-citation of paper that has the same plot
Now first by brute force search we know the minima for a single sodium atom.

%\begin{figure}[h!]
%	\centering
%	\includegraphics[scale=0.5]{./Inhalt/Bilder/optimal_defect_brute_force.png}
%	\caption{Brute force for single sodium atom}
%	\label{fig:bruteforcesodium}
%\end{figure}

\begin{figure}[h!]
	\centering
	\includegraphics[scale=0.5]{./Inhalt/Bilder/optimal_defect_simulated_annealing.png}
	\caption{Simulated Annealing for single sodium atom inserted, compared to a brute force minima search for a fixed number of removed atoms S.}
	\label{fig:simulatedannealingsodium}
\end{figure}  

So \ref{fig:simulatedannealingsodium} shows the simulated annealing algorithm accurately icks up on the location of the minima, albeit with no quantitative measure misleading local minima being blown out of proportion (See S=8).

\section{Sodium Dimer}
-3rd nearest neighbors, why ? because minima is more the 2 shells
-rough estimate of time used by brute forcing\\
-even with full cubic symmetry this is too long (divide b 64)\\
-maybe think about symmetry\\
-same graphics as for the monomer case\\
-heuristically take the maxima of these plots \\
-give structures\\
-discuss structures (inner shell carved out)\\

\begin{figure}
	\centering
	\includegraphics[scale = 0.5]{Inhalt/Bilder/optimal_defect_simulated_annealing_dimer.png}
	\caption{Simulated annealing results and their relative occurrence after 33 sweeps}
	\label{fig:simulatedannealingsodiumdimer}
\end{figure}

\chapter{DFT optical spectra results}
\label{chap:Zweites Kapitel}
%
Zweites Kapitel
%
%
\chapter{Conclusion}
Unfortunately no confidence or error can be estimated since the approach was purley heuristcal. Optical spectrum does not depend much on replacements and precise structure.



